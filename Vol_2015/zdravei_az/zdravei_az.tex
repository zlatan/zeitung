\begin{multicols}{2}
\byline{\sc\Large ЗДРАВЕЙ, АЗ}{Чавдар Йорданов, випуск, 1965г.}

\noindent \lettrine[lraise=0.2, nindent=0em, slope=-.5em]{Т}{ази} година се навършват 55 години от създаването на Немската езикова гимназия в 
Бургас. Като се замисля, това са страшно много години. Имах щастието да съм един 
от първия випуск на гимназията. Аз и моите съученици сега сме 68-годишни. 
Навремето хората на тази възраст ми изглеждаха ужасно стари. И не можех да си 
представя, че ще доживея до нея. А сега съм точно два пъти по-възрастен от моя 
баща...
\includegraphics[width=3.4in]{./zdravei_az/2-1.jpg}
Випускът ни беше от 80 човека, от които 50 момичета и 30 момчета, разпределени в 
4 класа. Бяхме от Бургас, от Варна, Хасково, Ямбол, Карнобат, Айтос, Пловдив, 
Сливен, Казанлък, Стара Загора и дори от София. Някъде прочетох, че този първи 
випуск за изминалите 55 години има най-висок успех в цялата история на 
гимназията.
\includegraphics[width=3.4in]{./zdravei_az/2-2.jpg}
От 1995 г. стана традиция ние, първите, да се събираме на всеки 5 години – в 
Бургас или София. От 15 години живеещите в София се събираме всяка последна 
сряда на месеца в едно заведение до първата спирка на трамвай No5, като през 
последните години започваха да идват и съученици от по-малките класове. Имаме 
чувството, че никога не сме се разделяли. Знаем си всичко за мъжете и жените, за 
децата и внуците, за болежките и радостите. И постоянно се връщаме към онези – 
първите 5 години...
\includegraphics[width=3.4in]{./zdravei_az/2-3.jpg}
\includegraphics[width=3.4in]{./zdravei_az/2-4.jpg}
\includegraphics[width=3.4in]{./zdravei_az/2-5.jpg}

\closearticle
\end{multicols}