\begin{multicols}{3}Учениците от Немската излъчват увереност и самочувствие
\byline{Интервю с г-жа Златина Янева}{Яна и Каролина, 10Д}

\noindent \includegraphics[width=2.1in]{./Jneva/Jneva.jpg}

Разкажете ни за себе си?

Казвам се Златина Янева. На 31 години от Бургас. Имам партньор до себе си и две годишно дете. Завършила съм Икономическия техникум (Търговската гимназия), една от първите експериментални паралелки с разширено изучаване на английски език. 
След това продължих образованието си в икономическия университет в град Варна, където завърших бакалавър и магистър счетоводство и контрол.Оттогава се занимавам със счетоводство и управлението на човешките ресурси. Обиколила съм няколко големи компании като Пикадили, аптеки "Марешки". Работила съм също така в сферата на международната логистика, като в момента жизненият ми път ме отведе 
отново в Бургас.

Ваше решение ли беше да се занимавате с икономика?

Да, изцяло мое решение беше, продиктувано от многото възможностите за развитие по онова време,както всеки от вас в момента търси най-изгодната професия, предлагаща повече възможности и по-високи доходи.

Защо решихте да кандидатствате за работа тук при нас?

По стечение на обстоятелствата се наложи да се върна обратно  в родния се град и реших да се развивам. Разглеждайки пазара на труда, ми се стори подходящо да пробвам нещо по-различно от това, което съм работила досега.

Какви са впечатленията ви от гимназията, учителския колектив и учениците?

До момента впечатленията ми са доста добри. Учителският колектив е сплотен, колегите са дружелюбни. Не ми е особено трудно да се адаптирам, понеже досега работата ми е била свързана с общуване с много хора.Трудности срещам само при запомнянето на имената(смее се). Също мога да кажа, че самото ниво на гимназията 
е високо, за което допринася не само учителският колектив, но и вие, учениците. Учениците излъчват увереност и самочувствие. Има голяма разлика в сравнение с другите училища. Средата понякога е важен фактор при бъдещото развитие.

Какво бихте желали да постигнете на този пост?

Преди всичко, ако мога да допринеса с нещо за по-доброто организиране на работата и разбира се да се развия в професионален план, защото всяко ново нещо, което човек научи, e полезно.

Разкажете за някоя интересна случка от ученическите ви години?

Нямаше мобилни телефони по онова време, от време на време се срещаха мобифони тухли четворки. Нямаше никакви възмости за комуникация. На една от срещите на бургаските клубове по дебати се бяхме събрали на една хижа в Карандила(Сливен). 
Водихме интересни и забавни дебати на различни теми. Вечерта преди да тръгнем беше навалял много сняг, преспите бяха около 3 метра, нас ни „изгониха” и освен непочистеният път се оказа, че има проблем с автобуса, понеже двигателят беше „умрял”(смее се). Около 40 човека трябвяше да чакаме от 9ч. сутринта  в преспите 
до вечерта към 22ч., когато на помощ ни се притече автобус тип 
„Чавдар”(раздрънкан автобус тип трабант). Имахме чувството, че ще се разпадне всеки момент, но в същото време бяхме толкова щастливи.

\closearticle
\end{multicols}
