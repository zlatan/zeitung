\begin{multicols}{3}В Немската има много таланти
\byline{Интервю с г-жа Мария Каравасилева}{Димитър Бодуров и Елена Андонова, 10Д}

Госпожа Мария Каравасилева започва работа в нашата гимназия преди 24 години. Традициите тогава са били свързани само с народна музика, но с времето е създаден танцовият състав през 1998 година, мажоретният състав през 2005 година и е възстановен фанфарният оркестър. За да ви предоставим отговорите на още интересни въпроси се срещнахме лично с нея. Ето част от интервюто ни:

\textit{Колко усилия са нужни и как протича организацията на един концерт или мюзикъл?}

Организацията на един концерт естествено не е работа за един човек, а трябва да се работи екипно. Учениците тогава се научават на отговорност, стриктност, на законите на сцената, защото ние нямаме професионални режисьори, но практиката ни научава. Много често се случва учениците сами да са си коменданти, но те се справят успешно, защото те вече са усвоили правилата. Един мюзикъл изисква хореографи, сценаристи, музиканти, озвучители и осветители, тоест покрай един такъв спектакъл учениците се научават и на други компетенции.

\textit{Достатъчни ли са часовете по музика?}

Преди години часовете по музика в 8 клас бяха два и имаше време да се пее и свири, а в 9 - един час, но сега и в 8, и в 9 клас има по един час, който е недостатъчен. Недостигът на часове поражда огромно желание и наплив от ученици, искащи да развият своя талант в изънкласното време. При момчетата липсата на часове в 10, 11 и 12 клас се отразява най-много, тъй като те тъкмо са преминали през мутацията на своите гласове и са готови да излязат на сцена.

Развитието на учениците най-често започва с училищния хор. Най-добрите от 8 клас преминават в концертния състав, който включва най-талантливите ученици  до 12 клас. След това се сформират вокални групи от двама-трима или солисти. Най-добрите стигат до върха на пирамидата и взимат главни роли, но също така има и изключения - деца, които получават съществена роля още в 8 клас, тъй като идват изградени като талант.

\textit{Какви са постигнатите ви успехи?}

Много ученици продължават да се занимават след завършването си с музика, макар че учат в езикова гимназия. Тя е първата им крачка към голямата сцена. Мога да дам много примери за деца, които тръгнали от мюзикълите, постигат истински успехи днес и в чужбина- Веселина Тодорова, която е завършила Московската академия с оперно пеене, Щерион Урумов, който е учил джаз китара в Швейцария. От последните години- Никола Парашкевов, който се насочи към театъра, но участваше и в мюзикъли, също и Мина Пенкова, член на група "Сезони". Всички те са доказателство, че немският език помага и отваря много врати в чужбина, а музиката за тях е прераснала от хоби в професия.
\closearticle
\end{multicols}
